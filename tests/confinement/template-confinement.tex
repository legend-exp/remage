
These tests check the generation of primary vertices by remage.
\subsection{Native sampling}
The first checks plot the positions for natively sampleable objects.



\begin{figure}[h!]
    \centering
    \includegraphics[width=0.45\textwidth]{native-surface.output_0000.png}
    \includegraphics[width=0.45\textwidth]{native-volume.output_0000.png}
    \caption{Geant4 visualisation of a simulation of the primary positions for the surface (left) and volume (right) of
    some natively sampled shapes. Primaries should be present on the G4Box (bottom row second from right),
    G4Orb (full sphere - above the box) and G4Tubs (left). The primaries should be on the detector surface or volume, although this
    is hard to appreciate with Geant4 visualisation.
    }
\end{figure}


Finally, we check the generation of points on complex objects.
\begin{figure}[h!]
    \centering
    \includegraphics[width=0.45\textwidth]{complex-surface.output_0000.png}
    \includegraphics[width=0.45\textwidth]{complex-volume.output_0000.png}

    \caption{Geant4 visualisation of a simulation of the primary positions for the surface (left) or volume (right) of
    some complex shapes. A polycone (top center), a box with a hole (center), the union of an orb and box (upper right)
    and an orb (right).
    }
\end{figure}



\subsection{Geometrical volumes and intersections}
The next tests check the generation of points in geometrical (user defined) volumes and intersections / unions of
geometrical and physical volumes.

\begin{figure}[h!]
    \centering
    \includegraphics[width=0.6\textwidth]{geometrical.output_0000.png}
    \caption{Geant4 visualisation of a simulation of the primary positions for some geometrical (user defined ) volumes.
    The primaries should be located in a sphere of center $(0,2,2)$ m and radius $0.7$ m (center overlapping box), a partially filled cylinder with center
    $(-2,-2,-2)$ m and outer radius $0.7$ m and height 1 m (far right) and a box of center $(0,2,-2)$ m and sides of length 0.7, 0.5 and 1.2 m (top).
    }
\end{figure}

\begin{figure}[h!]
    \centering
    \includegraphics[width=0.6\textwidth]{geometrical-or-physical.output_0000.png}
    \caption{Geant4 visualisation of a simulation of the primary positions for the Union of physical volumes defined
    by the union of the box and orb (upper left) and the partially filled sphere (lower center) and a geometric volume defined by
    sphere of center $(0,2,2)$ m and radius $0.7$ m near the center overlapping the box.
    The primaries should be in these three locations.}
\end{figure}


\begin{figure}[h!]
    \centering
    \includegraphics[width=0.6\textwidth]{geometrical-and-physical.output_0000.png}
    \caption{Geant4 visualisation of a simulation of the primary positions for the intersection of
    physical volumes defined by a user defined sphere of center $(1.5,2,2)$ m and radius $0.4$ m and the orb in the lower right.
    The primaries should be located in a subset of this orb.}
\end{figure}


\subsection{Union and intersection tests}
The next test simulates $2\nu\beta\beta$ decay in a hypothetical HPGe array and makes a
statistical comparison of the number of primaries in each volume.

\begin{figure}[h!]
    \centering
    \includegraphics[width=0.9\textwidth]{relative-ge.output.pdf}
    \caption{Statistical check on the number of primaries in each HPGe detector.}
\end{figure}


To check the intersection we generate events inside a cylinder around the HPGe detectors and again check the
positions and relative fractions.

\begin{figure}[h!]
    \centering
    \includegraphics[width=0.8\textwidth,page=1]{lar-in-check.output.pdf}
    \includegraphics[width=0.8\textwidth,page=2]{lar-in-check.output.pdf}
    \includegraphics[width=0.8\textwidth,page=3]{lar-in-check.output.pdf}
    \caption{Check on intersections by generating events in a cylinder around each HPGe string.}
\end{figure}

Finally check subtraction by generating events excluding these regions.
\begin{figure}[h!]
    \centering
    \includegraphics[width=0.45\textwidth,page=1]{lar-sub-check.output.pdf}
    \includegraphics[width=0.45\textwidth,page=1]{lar-int-and-sub-check.output.pdf}
    \includegraphics[width=0.45\textwidth,page=2]{lar-sub-check.output.pdf}
    \includegraphics[width=0.45\textwidth,page=2]{lar-int-and-sub-check.output.pdf}

    \caption{Check the vertex positions for a physical volume (the LAr) and subtracted geometrical volumes (cylinders) round each string.
    Left figures are just this subtraction while on the right we also intersect with a geometrically (user defined) cylinder the height of the 
    subtracted cylinders.
    }

\end{figure}

\subsection{Generic surface sampling}
The final part of this report describes checks on the generic surface sampler algorithm.
First check the bounding spheres for some solids.

\begin{figure}[h!]
    \centering
    \includegraphics[width=0.3\textwidth]{surface-sample-bounding-box-simple.output_0000.png}
    \includegraphics[width=0.3\textwidth]{surface-sample-bounding-box-subtraction.output_0000.png}
    \includegraphics[width=0.3\textwidth]{surface-sample-bounding-box-union.output_0000.png}

    \caption{Check the bounding sphere (red) for generating surface events in different volumes, left a cylinder, center the subtraction of two cylinders and left the union of two boxes. We also show the initial points of the lines (used to find intersections).
    The bounding sphere should fully contain the solid and the points should be outside (hard to appreciate).
    }
\end{figure}


\end{multicols}

Next, we generate surface events in various shapes and check the location of the primaries.

% check all the files exist

\begin{figure}[h!]
    \centering
    \includegraphics[width=0.45\textwidth]{vis-surface-trd.output_0000.png}
    \includegraphics[width=0.45\textwidth]{vis-surface-box.output_0000.png}
    \includegraphics[width=0.45\textwidth]{vis-surface-tubby.output_0000.png}
    \includegraphics[width=0.45\textwidth]{vis-surface-uni.output_0000.png}
    \includegraphics[width=0.45\textwidth]{vis-surface-sub.output_0000.png}
    \includegraphics[width=0.45\textwidth]{vis-surface-con.output_0000.png}
    \caption{Check the location of primary vertices for points generated on the surface of various shapes.
     The primaries should be generated in one solid at a time with the same layout as the figures are arranged.   }
\end{figure}


We make a more detailed test in python by extracting the coordinates of generated primaries and checking that they are close to a surface and then finding which.

% check all the files exist

\begin{figure}[h!]
    \centering
    \includegraphics[width=0.45\textwidth,page=1]{confinement.simple-solids-surface-trd.output.pdf}
    \includegraphics[width=0.45\textwidth,page=1]{confinement.simple-solids-surface-box.output.pdf}
    \includegraphics[width=0.45\textwidth,page=1]{confinement.simple-solids-surface-tubby.output.pdf}
    \includegraphics[width=0.45\textwidth,page=1]{confinement.simple-solids-surface-uni.output.pdf}
    \includegraphics[width=0.45\textwidth,page=1]{confinement.simple-solids-surface-sub.output.pdf}
    \includegraphics[width=0.45\textwidth,page=1]{confinement.simple-solids-surface-con.output.pdf}
    \caption{Check of the location of primary vertices for points generated on the surfaces of various shapes. The position of the vertices is plotted in python colouring by face each corresponds to.
    You should be able to see that the primaries are generated on the surface and the coloruing should be correct, although the perspective of the 3D plots makes this hard to appreciate.
        }
\end{figure}

% check all the files exist

\begin{figure}[h!]
    \centering
    \includegraphics[width=0.45\textwidth,page=2]{confinement.simple-solids-surface-trd.output.pdf}
    \includegraphics[width=0.45\textwidth,page=2]{confinement.simple-solids-surface-box.output.pdf}
    \includegraphics[width=0.45\textwidth,page=2]{confinement.simple-solids-surface-tubby.output.pdf}
    \includegraphics[width=0.45\textwidth,page=2]{confinement.simple-solids-surface-uni.output.pdf}
    \includegraphics[width=0.45\textwidth,page=2]{confinement.simple-solids-surface-sub.output.pdf}
    \includegraphics[width=0.45\textwidth,page=2]{confinement.simple-solids-surface-con.output.pdf}

    \caption{Check the location of primary vertices for points generated on the surfaces of various shapes, showing 2D projections onto the (x,y), (r,z) and (x,z) planes. The position of the vertices is plotted in python colouring by face each corresponds to.
    You should be able to see that the primaries are generated on the surface and the coloring should be correct.
        }
\end{figure}


Finally, we count the number of events on each face and make a statical test that this is compatible with the surface area.

\begin{figure}[h!]
    \centering
    \includegraphics[width=0.45\textwidth,page=3]{confinement.simple-solids-surface-trd.output.pdf}
    \includegraphics[width=0.45\textwidth,page=3]{confinement.simple-solids-surface-box.output.pdf}
    \includegraphics[width=0.45\textwidth,page=3]{confinement.simple-solids-surface-tubby.output.pdf}
    \includegraphics[width=0.45\textwidth,page=3]{confinement.simple-solids-surface-uni.output.pdf}
    \includegraphics[width=0.45\textwidth,page=3]{confinement.simple-solids-surface-sub.output.pdf}
    \includegraphics[width=0.45\textwidth,page=3]{confinement.simple-solids-surface-con.output.pdf}

    \caption{Fraction of the primary vertices located on each face of the various shapes. This is compared to the expectation
    that the fraction is proportional to the surface area. The lower plot shows a residual. These points should be compatible with the expectation with only some statistical scatter.
        }
\end{figure}
